\documentclass{article}

\usepackage[final]{packages/neurips_2020}

\usepackage{biblatex}
\addbibresource{bibliography.bib}

\usepackage[utf8]{inputenc} % allow utf-8 input
\usepackage[T1]{fontenc}    % use 8-bit T1 fonts
\usepackage{hyperref}       % hyperlinks
\usepackage{url}            % simple URL typesetting
\usepackage{booktabs}       % professional-quality tables
\usepackage{amsfonts}       % blackboard math symbols
\usepackage{nicefrac}       % compact symbols for 1/2, etc.
\usepackage{microtype}      % microtypography

\title{Device-Friendly Privacy-Preserving Generative Adversarial Networks}

% The \author macro works with any number of authors. There are two commands
% used to separate the names and addresses of multiple authors: \And and \AND.
%
% Using \And between authors leaves it to LaTeX to determine where to break the
% lines. Using \AND forces a line break at that point. So, if LaTeX puts 3 of 4
% authors names on the first line, and the last on the second line, try using
% \AND instead of \And before the third author name.

\author{
    Sergio Rivera \\
    Department of Computer Science \\
    University of Cambridge \\
    \texttt{sr2070@cam.ac.uk}
}

\begin{document}

    \maketitle

    \begin{abstract}
        The abstract paragraph should be indented \nicefrac{1}{2}~inch (3~picas) on
        both the left- and right-hand margins. Use 10~point type, with a vertical
        spacing (leading) of 11~points.  The word \textbf{Abstract} must be centered,
        bold, and in point size 12. Two line spaces precede the abstract. The abstract
        must be limited to one paragraph.
    \end{abstract}

    \section{Background}

    \subsection{Generative Adversarial Networks}

    First announced in 2014, Generative Adversarial Networks\cite{goodfellowGenerativeAdversarialNetworks2014} introduced the possibility of learning entire data
    distributions in a fully unsupervised manner by employing concepts of adversarial training and generative modelling.
    Over the years, GANs have proven to be useful in a wide variety of applications, ranging from simulating costly
    experiments in particle physics[^2] to turning horses into zebras by applying image translation techniques[^3].
    Other interesting works have successfully applied GANs to image synthesis[^5] and photo-realistic
    super-resolution[^6].

    \subsection{Footnotes}

    \subsection{Figures}

    \begin{figure}
        \centering
        \fbox{\rule[-.5cm]{0cm}{4cm} \rule[-.5cm]{4cm}{0cm}}
        \caption{Sample figure caption.}
    \end{figure}

    \subsection{Tables}

    \begin{table}
        \caption{Sample table title}
        \label{sample-table}
        \centering
        \begin{tabular}{lll}
            \toprule
            \multicolumn{2}{c}{Part}                   \\
            \cmidrule(r){1-2}
            Name     & Description     & Size ($\mu$m) \\
            \midrule
            Dendrite & Input terminal  & $\sim$100     \\
            Axon     & Output terminal & $\sim$10      \\
            Soma     & Cell body       & up to $10^6$  \\
            \bottomrule
        \end{tabular}
    \end{table}

    \printbibliography

\end{document}